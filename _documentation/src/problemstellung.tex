% Problemstellung und entwickelte Lösungen

\subsection{Das Grundproblem}

Performance-Künstler wollen Bewegungen in Echtzeit in Visualisierungen umwandeln. Das klingt einfach, ist aber technisch knifflig: 

\textbf{Konkrete Herausforderungen:}
\begin{itemize}
    \item \textbf{Beleuchtungskonflikt:} Beamerprojektion überblendet RGB-Kameras
    \item \textbf{Partielle Sichtbarkeit:} Tänzer verdecken oft Körperteile - Standard-Kinect verliert dann das Skelett
    \item \textbf{Timing:} Live-Performance braucht sofortige Reaktion ohne spürbare Verzögerung
    \item \textbf{Setup-Aufwand:} Jede neue Kamera-/Beamerposition erfordert stundenlange Neukalibrierung
\end{itemize}



\textbf{Projektkontext 'Echoes of the Mind':}
Die Filmakademie Ludwigsburg benötigte für ihre Tanzproduktion:
\begin{itemize}
    \item Bodenprojektion mit Tracking von oben
    \item Frontale sowie von hinten auf die Kamera leuchtende Beamerbeleuchtung mit gleichzeitigem Tracking
    \item Daraus resultierend: Drei verschiedene Visualisierungsmodi
    \item Integration des Trackings in die drei bestehenden TouchDesigner-Workflows
\end{itemize}

\subsection{Entwickelte Lösungsansätze}

\textbf{Infrarot statt RGB:}
Wir umgingen das Beleuchtungsproblem, indem wir den Infrarot-Stream der Kinect V2 über OBS Virtual Camera an MediaPipe weiterleiten. Infrarot funktioniert unabhängig von sichtbarer Beleuchtung.

\textbf{MediaPipe für robusteres Tracking:}
MediaPipes ML-Modell kommt besser mit partiell verdeckten Körpern zurecht als die native Kinect-Skelett-Erkennung. Bei Tests mit 50\% Körperokklusion blieb MediaPipe stabil, während Kinect das Skelett verlor.

\textbf{64-Spike-Visualisierung:}
Ein radialer Kreis mit 64 Segmenten reagiert auf Hand- und Fußbewegungen. Mit 5,625° Winkelauflösung ist das System präzise genug für subtile choreographische Gesten.

\textbf{Live-Kalibrierung:}
Ein Parametersystem in hervorgehobenen TouchDesigner Nodes ermöglicht X/Y-Verschiebung, Rotation und Skalierung während des Betriebs. Setup-Zeit: von 2-3 Stunden auf unter 30 Minuten.

\textbf{Modularer Aufbau:}
Tracking, Analyse und Visualisierung sind getrennte TouchDesigner-Container. Das ermöglicht parallele Entwicklung und einfache Integration in bestehende Projekte.

\subsection{Was dabei herauskam}

\textbf{Einbindung von Körperpositions-Daten in Visuals:}
Durch 7 Python Skripte wurden Daten gesammelt, Distanz und Winkel berechnet, bereitgestellt, und in verschiedene Node-Strukturen zu verschiedenen Zwecken eingebunden. Darunter particleGPU, Switches, SOPs etc.

\textbf{Funktionierender Infrarot-Workflow:}
Die Kombination aus Kinect V2 IR-Stream und MediaPipe löst das Beleuchtungsproblem zuverlässig. Das war nicht selbstverständlich - da MediaPipe für RGB-Daten optimiert ist.

\textbf{Lernerkenntnisse:}
Der ursprünglich geplante Kalman-Filter für Kinect-MediaPipe-Fusion erwies sich als unnötig komplex. MediaPipe allein ist für unsere Anwendung ausreichend robust.

\subsection{Einordnung}

M.A.S.K. ist kein Ersatz für professionelle Motion-Capture-Systeme wie OptiTrack. Es füllt die Lücke zwischen Consumer-Hardware und Profi-Equipment für Live-Performance unter schwierigen Lichtverhältnissen. Die Lösung funktioniert mit Standard-Hardware (Komplettpaket 90€ Kinect V2 mit Adaptern, günstiger Second-Hand) und kostenloser Software.