Die Entwicklung des M.A.S.K.-Systems erforderte systematisches Risikomanagement aufgrund der interdisziplinären Natur des Projekts, der festen Produktionsdeadline und der Abhängigkeit von verschiedenen Hardware- und Software-Komponenten.

\subsection{Identifizierte Risikokategorien}

\textbf{Hardware-Abhängigkeiten:}
\begin{itemize}
    \item \textbf{Risiko:} Kinect V2 Hardware-Ausfälle oder Kompatibilitätsprobleme
    \item \textbf{Wahrscheinlichkeit:} Mittel
    \item \textbf{Auswirkung:} Hoch - Projektstillstand
    \item \textbf{Mitigation:} Beschaffung von Backup-Kinect V2, Tests verschiedener Hardware-Konfigurationen
\end{itemize}

\textbf{Software-Integration:}
\begin{itemize}
    \item \textbf{Risiko:} Inkompatibilität zwischen MediaPipe, TouchDesigner und Kinect SDK
    \item \textbf{Wahrscheinlichkeit:} Hoch - unterschiedliche Update-Zyklen
    \item \textbf{Auswirkung:} Hoch - Grundlegende Systemarchitektur betroffen
    \item \textbf{Mitigation:} Frühe Proof-of-Concept-Implementierung, Version-Locking kritischer Komponenten
\end{itemize}

\textbf{Performance-Risiken:}
\begin{itemize}
    \item \textbf{Risiko:} Unzureichende Echtzeit-Performance für Live-Anwendung
    \item \textbf{Wahrscheinlichkeit:} Mittel
    \item \textbf{Auswirkung:} Hoch - System unbrauchbar für Produktion
    \item \textbf{Mitigation:} Kontinuierliche Performance-Tests, GPU-Optimierung, Fallback-Modi
\end{itemize}

\textbf{Zeitplan-Risiken:}
\begin{itemize}
    \item \textbf{Risiko:} Verzögerungen durch unvorhergesehene technische Probleme
    \item \textbf{Wahrscheinlichkeit:} Hoch - Forschungscharakter des Projekts
    \item \textbf{Auswirkung:} Kritisch - Feste Produktionsdeadline
    \item \textbf{Mitigation:} Agile Sprint-Struktur, kontinuierliche Stakeholder-Kommunikation, Scope-Reduzierung bei Bedarf
\end{itemize}

\textbf{Kommunikationsrisiken:}
\begin{itemize}
    \item \textbf{Risiko:} Missverständnisse zwischen technischer und künstlerischer Seite
    \item \textbf{Wahrscheinlichkeit:} Mittel
    \item \textbf{Auswirkung:} Mittel - Entwicklung in falsche Richtung
    \item \textbf{Mitigation:} Regelmäßige Demos, gemeinsame Prototyping-Sessions, klare Dokumentation
\end{itemize}

\textbf{Umgebungsabhängigkeiten:}
\begin{itemize}
    \item \textbf{Risiko:} Unvorhersehbare Lichtverhältnisse im Produktionsstudio
    \item \textbf{Wahrscheinlichkeit:} Hoch
    \item \textbf{Auswirkung:} Mittel - Tracking-Qualität beeinträchtigt
    \item \textbf{Mitigation:} Infrarot-basierte Lösung, Kalibrierungs-Tools, Backup-Beleuchtungsszenarien
\end{itemize}

\textbf{Equipment-Integration:}
\begin{itemize}
    \item \textbf{Risiko:} Inkompatibilität mit vorhandener Studio-Infrastruktur
    \item \textbf{Wahrscheinlichkeit:} Mittel
    \item \textbf{Auswirkung:} Hoch
    \item \textbf{Mitigation:} Vorab-Tests im Studio, modulare Systemarchitektur, Standard-Schnittstellen
\end{itemize}

\subsection{Implementierte Risikominderungsstrategien}

\textbf{Modularität als Risikominderung:}
\raggedright Die containerisierte TouchDesigner-Architektur ermöglichte es, einzelne Komponenten bei Problemen zu ersetzen oder zu modifizieren, ohne das Gesamtsystem zu gefährden.

\textbf{Dual-Source-Tracking-Konzept:}
\raggedright Die ursprüngliche Planung eines Hybrid-Systems aus Kinect und MediaPipe bot Redundanz bei Hardware-Ausfällen. Auch nach der Fokussierung auf MediaPipe blieb die Kinect als Backup-Option verfügbar.

\textbf{Infrarot-Adaptation:}
\raggedright Die Entwicklung der infrarotbasierten Pipeline löste das kritische Risiko der Beleuchtungsinterferenz und machte das System robust gegenüber variablen Lichtverhältnissen.

\textbf{Sprint-basierte Entwicklung:}
\raggedright Die Aufteilung in sechs Sprints ermöglichte frühe Identifikation von Problemen und kontinuierliche Anpassung der Projektziele:

\begin{itemize}
    \item Sprint 1-2: Risiko-Assessment und Machbarkeitsvalidierung
    \item Sprint 3-4: Kritische Pfad-Implementierung
    \item Sprint 5-6: Produktionsoptimierung und Fallback-Entwicklung
\end{itemize}

\textbf{Kontinuierliche Stakeholder-Integration:}
\raggedright Regelmäßige Demos und Feedback-Schleifen verhinderten Fehlentwicklungen und stellten sicher, dass technische Lösungen den künstlerischen Anforderungen entsprachen.

\subsection{Eingetretene Risiken und Bewältigung}

\textbf{Kinect-MediaPipe Datenfusion:}
\textbf{Problem:} \raggedright Die ursprünglich geplante Kalman-Filter-basierte Datenfusion erwies sich als zu komplex für die verfügbare Entwicklungszeit.

\textbf{Bewältigung:} \raggedright Pivot zu MediaPipe-only Implementierung in Sprint 4, wodurch Komplexität reduziert und Robustheit verbessert wurde.

\textbf{Lessons Learned:} \raggedright Einfachere Lösungen sind oft robuster; komplexe Datenfusion nur bei klarem Mehrwert implementieren.

\textbf{Beleuchtungsinterferenz:}
\textbf{Problem:} RGB-Tracking versagte bei intensiver Beamer-Beleuchtung auf Performer.

\textbf{Bewältigung:} Entwicklung der infrarotbasierten Pipeline über OBS Virtual Camera in Sprint 6.

\textbf{Lessons Learned:} \raggedright Hardware-Limitierungen erfordern oft kreative Software-Lösungen; alternative Sensor-Modalitäten erkunden.

\textbf{Produktions-Setup-Komplexität:}
\textbf{Problem:} Kalibrierung bei verschiedenen Kamera-Beamer-Ausrichtungen zeitaufwändiger als erwartet.

\textbf{Bewältigung:} Entwicklung parametrischer Offset-Systeme für Echtzeit-Kalibrierung.

\textbf{Lessons Learned:} \raggedright Produktionsumgebungen sind unvorhersehbar; flexible Kalibrierungs-Tools essentiell.

\subsection{Nicht eingetretene Risiken}

\textbf{Hardware-Ausfälle:}
\raggedright Trotz intensiver Nutzung blieb die Kinect V2 Hardware während der gesamten Projektlaufzeit stabil. Die Backup-Hardware wurde nicht benötigt.

\textbf{Fundamentale Software-Inkompatibilitäten:}
\raggedright Die Kombination aus TouchDesigner, MediaPipe und Kinect SDK erwies sich als stabil. Version-Locking verhinderte Probleme durch automatische Updates.

\textbf{Performance-Limitierungen:}
\raggedright Das entwickelte System erreichte die erforderliche Echtzeit-Performance ohne größere Optimierungen. GPU-Shader-Integration bot zusätzliche Leistungsreserven.

\subsection{Risikomanagement-Erkenntnisse}

\textbf{Erfolgreiche Strategien:}
\begin{itemize}
    \item \textbf{Frühe Prototyping:} Schnelle Validierung kritischer Annahmen
    \item \textbf{Modulare Architektur:} Isolierung von Risiken auf Komponenten-Ebene
    \item \textbf{Kontinuierliche Kommunikation:} Frühe Identifikation von Problemen
    \item \textbf{Flexible Zielsetzung:} Anpassung des Scope bei technischen Constraints
\end{itemize}

\textbf{Verbesserungspotential:}
\begin{itemize}
    \item \textbf{Umfangreichere Vorab-Tests:} Mehr Zeit für Studio-Integration-Tests
    \item \textbf{Backup-Szenarien:} Detailliertere Fallback-Pläne für kritische Komponenten
    \item \textbf{Externe Dependencies:} Bessere Evaluation von Third-Party-Risiken
\end{itemize}

\textbf{Risiko-Assessment-Framework:}
Für ähnliche interdisziplinäre Technologieprojekte wird empfohlen:

\begin{enumerate}
    \item \textbf{Technology-Risk-Matrix:} Systematische Bewertung aller technischen Dependencies
    \item \textbf{Stakeholder-Risk-Analysis:} Berücksichtigung verschiedener Risikotoleranz-Level
    \item \textbf{Continuous Risk Monitoring:} Regelmäßige Neubewertung während Projektlaufzeit
    \item \textbf{Mitigation-Prototyping:} Frühe Entwicklung kritischer Fallback-Lösungen
\end{enumerate}

Das systematische Risikomanagement erwies sich als entscheidender Erfolgsfaktor für M.A.S.K. Die Kombination aus proaktiver Risikoidentifikation, flexibler Projektorganisation und technischer Redundanz ermöglichte die erfolgreiche Bewältigung aller kritischen Herausforderungen.