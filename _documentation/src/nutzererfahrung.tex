Die Evaluation des M.A.S.K.-Systems erfolgte durch kontinuierliches Feedback der Stakeholder während der Entwicklungsphase sowie durch direkte Beobachtung während der Produktionsvalidierung im Albrecht-Ade-Studio.

\subsection{Stakeholder-Feedback während der Entwicklung}

\textbf{Filmakademie-Team Evaluation:}
\raggedright Das Projektteam der Filmakademie Baden-Württemberg bewertete M.A.S.K. durchweg positiv in Bezug auf die technische Umsetzung der künstlerischen Vision. Besonders hervorgehoben wurde:

\begin{itemize}
    \item \textbf{Responsivität der Visualisierungen:} \raggedright Die Echtzeit-Reaktion der Visuals auf Körperbewegungen entsprach den choreographischen Anforderungen
    \item \textbf{Flexibilität des Systems:} \raggedright Die Möglichkeit, zwischen verschiedenen Visualisierungsmodi zu wechseln, erwies sich als entscheidend für die filmische Umsetzung
    \item \textbf{Zuverlässigkeit unter Produktionsbedingungen:} Das System funktionierte stabil während der gesamten Dreharbeiten
\end{itemize}

\textbf{Choreograph und Performer Feedback:}
\raggedright Der Tänzer berichtete von einer natürlichen Interaktion mit dem System, wobei die Visualisierungen die emotionale Ausdruckskraft der Choreographie verstärkten. Kritische Punkte:

\begin{itemize}
    \item \textbf{Lernkurve:} Anfängliche Anpassung an die responsive Umgebung erforderlich
    \item \textbf{Bewegungsfreiheit:} \raggedright Das System ermöglichte natürliche Bewegungen ohne Einschränkungen durch Marker oder Sensoren
    \item \textbf{Visuelle Integration:} Die Projektionen fühlten sich als natürliche Erweiterung der Körperbewegung an
\end{itemize}

\textbf{Setup und Bedienbarkeit:}
Das Kamerateam und die technische Crew bewerteten verschiedene Aspekte der Systemintegration:

\textbf{Positive Aspekte:}
\begin{itemize}
    \item Schnelle Einrichtung der Grundkonfiguration
\end{itemize}

\textbf{Verbesserungspotential:}
\begin{itemize}
    \item Kalibrierungsprozess bei Kamera-Perspektivwechseln zeitaufwändig
    \item Beleuchtungsänderungen erfordern manuelle Anpassungen
\end{itemize}

\subsection{Produktionsvalidierung}

\textbf{Leistung nach Visualisierungsmodus:}

\textbf{Visual 1 - Hand-Feuer-Effekte:}
\begin{itemize}
    \item Ausgezeichnete Performance nach initialem Setup
    \item Präzise Verfolgung der Handbewegungen
    \item Stabile Partikel-Generierung über gesamte Aufnahmedauer
\end{itemize}

\textbf{Visual 2 - Adaptive Kopfpartikel:}
\begin{itemize}
    \item Herausforderungen bei frontaler Kameraposition
    \item Erfolgreiche Echtzeit-Kalibrierung während Aufnahmen
    \item Zufriedenstellende Endergebnisse trotz technischer Komplexität
    \item Durch Filmkamerawinkel viel Offset-setzen in Absprache mit Filmcrew
\end{itemize}

\textbf{Visual 3 - Radiales Spike-System:}
\begin{itemize}
    \item Fehlerfreie Performance bei Top-Down-Konfiguration
    \item Hohe Präzision des 64-Spike-Systems
    \item Optimale Integration mit infrarotbasiertem Tracking
\end{itemize}

\subsection{Vergleich mit Projektalternativen}

\textbf{Gegenüberstellung zu kommerziellen Lösungen:}
Die Stakeholder verglichen M.A.S.K. mit theoretischen Alternativen:

\textbf{Vorteile gegenüber professionellen Motion-Capture-Systemen:}
\begin{itemize}
    \item Deutlich geringere Kosten (Kinect V2 vs. OptiTrack-Setup)
    \item Keine Marker-Platzierung erforderlich
    \item Flexiblere Beleuchtungsbedingungen durch IR-Tracking
    \item Einfachere Integration in bestehende TouchDesigner-Workflows
\end{itemize}

\textbf{Limitierungen im Vergleich:}
\begin{itemize}
    \item Geringere absolute Präzision bei Millimeter-genauen Anforderungen
    \item Abhängigkeit von Sichtlinie zum Sensor
    \item Begrenzte Tracking-Reichweite
\end{itemize}

\subsection{Lessons Learned aus Stakeholder-Sicht}

\textbf{Projektmanagement-Erkenntnisse:}
\textbf{Erfolgreiche Aspekte:}
\begin{itemize}
    \item Regelmäßige Kommunikation zwischen beiden Standorten
    \item Iterative Entwicklung ermöglichte Anpassungen an künstlerische Vision
    \item Frühe Prototypen halfen bei Konzeptvalidierung
\end{itemize}

\textbf{Interdisziplinäre Zusammenarbeit:}
\raggedright Die Kooperation zwischen technischer Entwicklung und künstlerischer Anwendung erwies sich als bereichernd für beide Seiten:

\begin{itemize}
    \item Technische Constraints führten zu kreativen choreographischen Lösungen
    \item Künstlerische Anforderungen führten zu praktischen technischen Lösungen
    \item Gemeinsame Problemlösung stärkte Projektverständnis
\end{itemize}

\subsection{Empfehlungen für zukünftige Implementierungen}

Basierend auf dem Stakeholder-Feedback wurden folgende Empfehlungen für zukünftige M.A.S.K.-Implementierungen entwickelt:

\textbf{Technische Verbesserungen:}
\begin{itemize}
    \item Automatisierte Kalibrierungsroutinen für verschiedene Kamera-Setups
    \item Erweiterte Konfidenz-basierte Tracking-Interpolation
    \item Integration zusätzlicher Sensor-Modalitäten für robusteres Tracking
\end{itemize}

\textbf{Workflow-Optimierungen:}
\begin{itemize}
    \item Preset-System für häufige Studio-Konfigurationen
    \item Verbesserte Debug-Interfaces für technische Crews
    \item Standardisierte Setup-Checklisten für Produktionsumgebungen
\end{itemize}

\textbf{Dokumentation und Training:}
\begin{itemize}
    \item Best-Practice-Dokumentation für verschiedene Anwendungsszenarien
\end{itemize}
