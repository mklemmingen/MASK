\subsection{Professionelle Kollaboration zwischen Hochschulen}

\textbf{Interdisziplinäres Entwicklungsteam:}

\textbf{Technische Führung - Hochschule Reutlingen:}
\begin{itemize}
    \item \textbf{Entwickler:} Marty Lauterbach (Sole Developer, Technische Projektleitung)
    \item \textbf{Rolle:} System-Architektur, MediaPipe-Integration, TouchDesigner-Pipeline-Entwicklung
    \item \textbf{Verantwortung:} Vollständige technische Implementation als einziger Entwickler der Hochschule
    \item \textbf{Entwicklung:} Eigenständige Umsetzung der Infrarot-MediaPipe-Pipeline
\end{itemize}

\textbf{Kreative Partner - Filmakademie Baden-Württemberg:}
\begin{itemize}
    \item \textbf{Designerinnen:} Maja Litzke und Rahel Fundinger (Creative Direction, Künstlerische Leitung)
    \item \textbf{Projektkontext:} "Echoes of the Mind" - Cinematographische Tanzproduktion über mentale Zustände
    \item \textbf{Kollaborationsqualität:} Intensive interdisziplinäre Partnerschaft zwischen Technik und Kunst
    \item \textbf{Kommunikation:} Regelmäßige Sprint-Meetings und kontinuierlicher fachlicher Austausch
\end{itemize}

\textbf{Interdisziplinäre Konstellation:}
Diese Kooperation zwischen einem Hochschulentwickler und erfahrenen Filmakademie-Designerinnen ermöglichte eine produktive Synthese aus technischer Entwicklung und künstlerischer Vision.

\textbf{Betreuung und Beratung:}

\textbf{Hochschule Reutlingen:}
\begin{itemize}
    \item \textbf{Projektbetreuung:} Anja Hartmann (Akademische Betreuung, Bewertung)
    \item \textbf{Fachberatung:} Niklas Zidarov Digital Art (Technische Beratung, 17.01.2025)
    \item \textbf{Rolle:} Akademische Unterstützung
\end{itemize}

\textbf{Produktionsumgebung:}
\begin{itemize}
    \item \textbf{Albrecht-Ade-Studio:} Produktionsstudio der Filmakademie
    \item \textbf{Technische Crew:} Kamerateam, Beleuchtung, Tontechnik
    \item \textbf{Equipment-Management:} Hardware-Bereitstellung und -Support
\end{itemize}

\subsection{Kommunikationsprotokoll}

\textbf{Meeting-Struktur:}

\textbf{Regelmäßige Termine:}
\begin{itemize}
    \item \textbf{Sprint-Meetings:} Alle 2-3 Wochen, 90 Minuten, Online mit Filmakademie
    \item \textbf{Demo-Sessions:} Nach jedem Sprint, 60 Minuten, Live-Demonstration der Fortschritte
    \item \textbf{Technical-Reviews:} Bei Bedarf, Problem-focused
    \item \textbf{Production-Meetings:} Pre-Production, intensivere Koordination für Studioeinsatz
\end{itemize}

\textbf{Laufende Kommunikation:}
\begin{itemize}
    \item \textbf{WhatsApp-Gruppe:} Schnelle Updates, Terminkoordination, Dateiaustausch
    \item \textbf{Video-Calls:} Spontane Problem-Lösung, Screen-Sharing für technische Diskussionen
    \item \textbf{GitHub-Repository:} Code-Sharing, Versions-Dokumentation
\end{itemize}

\subsection{Professionelle Kollaborationsmeilensteine}

\textbf{Technische Grundlegung (17.01.2025):}
\textbf{Ausgangssituation:} Als einziger Entwickler der Hochschule Reutlingen suchte ich Expertenberatung für eine geeignete Motion-Capture-Strategie.
\begin{itemize}
    \item \textbf{Fachberatung:} Lehrbeauftragter Digital Art empfahl MediaPipe-Integration
    \item \textbf{Strategische Entscheidung:} MediaPipe als Subprozess für Kinect RGB-Stream
    \item \textbf{Technische Vision:} Hybrid-System für verbesserte Tracking-Robustheit
    \item \textbf{Eigenständige Umsetzung:} Vollständige Implementation-Verantwortung übernommen
\end{itemize}

\textbf{Interdisziplinärer Projekt-Kick-off (Ende Januar 2025):}
\textbf{Erste Zusammenarbeit:} Intensive Kollaboration mit Maja Litzke und Rahel Fundinger zur Definition der technischen Architektur basierend auf künstlerischen Anforderungen.

\textbf{Gemeinsam entwickelte Systemarchitektur:}
\begin{itemize}
    \item \textbf{Dual-Source-Konzept:} MediaPipe-Kinect-Fusion basierend auf technischer Analyse und kreativen Anforderungen
    \item \textbf{Interdisziplinäre Integration:} Choreographie-spezifische Trigger-Pipeline durch kollaborative Spezifikation
    \item \textbf{Bewegungsanalyse:} Geschwindigkeit, Rotation und Winkelbeziehungen entsprechend künstlerischer und technischer Kriterien
    \item \textbf{Kollaborative Arbeitsweise:} Sprint-basierte Entwicklung mit regelmäßigen interdisziplinären Abstimmungen
\end{itemize}

\textbf{Kreative Anforderungsdefinition (10.03.2025):}
\textbf{Wendepunkt der Zusammenarbeit:} Die Designerinnen präsentierten konkrete choreographische Konzepte, die meine technischen Lösungsansätze fundamental beeinflussten.

\textbf{Interdisziplinäre Systemdefinition:}
\begin{enumerate}
    \item \textbf{Hand-Tracking-Priorisierung:} Fokus auf Arm- und Handbewegungen nach kreativer Spezifikation
    \item \textbf{Räumliche Kalibrierung:} Präzise Beamer-Projektion für Bodenchoreographie
    \item \textbf{Visual-Responsivität:} Echtzeit-Bewegungserkennung für emotionale Tanznarrative  
    \item \textbf{Multi-Perspektiven-System:} Top-Down und frontale Tracking-Modi
\end{enumerate}

\textbf{Interdisziplinäre Translation:} Künstlerische Anforderungen und technische Möglichkeiten wurden gemeinsam in umsetzbare Implementation-Spezifikationen entwickelt.

\subsection{Entscheidungslogprotokoll}

\textbf{Eigenständige Architekturentscheidungen:}

\textbf{Hybrid-Tracking-System Definition (Ende Januar 2025)}
\begin{itemize}
    \item \textbf{Kontext:} Projekt-Kick-off mit Filmakademie zur Systemarchitektur
    \item \textbf{Entwickler-Entscheidung:} Dual-Source-Ansatz basierend auf Technical Review
    \item \textbf{Begründung:} Kombination aus MediaPipe-Robustheit und Kinect-Tiefendaten
    \item \textbf{Implementation-Plan:} Kalman-Filter-basierte Datenfusion
\end{itemize}

\textbf{MediaPipe-Priorisierung (Sprint 2, bis 24.03.2025)}
\begin{itemize}
    \item \textbf{Eigenständige Evaluation:} Komparative Tests zeigen MediaPipe-Überlegenheit bei partieller Okklusion
    \item \textbf{Stakeholder-Konsultation:} Filmakademie bevorzugt Robustheit über absolute Präzision
    \item \textbf{Technische Analyse:} MediaPipe höhere Confidence-Werte bei nicht-sichtbaren Körperregionen
    \item \textbf{Entwicklungsentscheidung:} MediaPipe als primäres System, Kinect als Backup
    \item \textbf{Dokumentation:} Testing\_KinectVsMediapipe.toe als Vergleichsreferenz
\end{itemize}

\textbf{Kinect-Deprecation (Sprint 4, bis 20.04.2025)}
\begin{itemize}
    \item \textbf{Problem-Identifikation:} Hybrid-System erhöht Komplexität ohne klaren Mehrwert
    \item \textbf{Stakeholder-Feedback:} Vereinfachung für bessere Wartbarkeit gewünscht
    \item \textbf{Performance-Analyse:} MediaPipe-Only zeigt gute Performance bei unvollständigen Skeleton-Daten
    \item \textbf{Architektur-Refactoring:} Vollständige Modularisierung von MediaPipe, Elimination der Kinect-Dependencies
    \item \textbf{Begründung:} Limitierte Partial-Body-Detection der Kinect vs. MediaPipe-Robustheit
\end{itemize}

\textbf{Iterative Scope-Entwicklung:}

\textbf{Bubble-Physics zu Visual-System-Trias (Sprint 4-5)}
\begin{itemize}
    \item \textbf{Ursprünglicher Plan:} Komplexe Bubble-Displacement-Algorithmen
    \item \textbf{Technische Herausforderung:} Unzureichende Feedback-Loops für persistente Bubble-Lifecycles
    \item \textbf{Evaluierte Implementierungen:} C\# Shader, ParticleGPU Manipulation, Hybrid CPU-GPU Approach
    \item \textbf{Stakeholder-Realignment:} Alternative Visual-Ansätze für praktische Choreographie-Integration
    \item \textbf{Finale Implementierung:} Hand-Fire-Effects, Adaptive Head-Particles, Radial Spike-System
    \item \textbf{Outcome:} Drei konkrete, choreographie-spezifische und produktionstaugliche Visualisierungen
\end{itemize}

\textbf{Infrarot-Lösung (Sprint 6, bis 12.05.2025)}
\begin{itemize}
    \item \textbf{Kritisches Problem:} Beamer-Projektionen verursachen RGB-Kamera-Überbelichtung
    \item \textbf{Stakeholder-Impact:} Fundamental für Produktionsumgebung der Filmakademie
    \item \textbf{Eigenständige Lösung:} Kinect V2 Infrarot-Stream als Primary Input über OBS Virtual Camera
    \item \textbf{Technische Implementation:} IR-optimierte MediaPipe Skeleton-Detection
    \item \textbf{Zeitmanagement:} Zusätzliche Entwicklungszeit für fundamentale Problemlösung
    \item \textbf{Validation:} Vollständige Beleuchtungs-Immunität in Produktionsumgebung erreicht
\end{itemize}

\subsection{Eigenständige Entwicklungsprozesse}

\textbf{Solo-Development-Workflow:}

\textbf{Tägliche Entwicklungsroutine:}
\begin{itemize}
    \item \textbf{Morgendliche Prioritätssetzung:} Sprint-Ziele vs. technische Herausforderungen
    \item \textbf{Iterative Implementation:} Kurze Entwicklungszyklen mit kontinuierlicher Validierung
    \item \textbf{Dokumentation parallel:} Code-Kommentierung und Architektur-Notizen während Entwicklung
    \item \textbf{Abendliche Reflektion:} Fortschritts-Assessment und nächste Schritte
\end{itemize}

\textbf{Problem-Solving-Ansatz:}
\begin{itemize}
    \item \textbf{Eigenständige Recherche:} MediaPipe-Dokumentation, TouchDesigner-Community
    \item \textbf{Prototyping:} Schnelle Proof-of-Concept-Implementierungen
    \item \textbf{Stakeholder-Konsultation:} Bei konzeptionellen oder künstlerischen Entscheidungen
    \item \textbf{Iterative Verfeinerung:} Kontinuierliche Optimierung basierend auf Testing
\end{itemize}

\textbf{Sprint-Retrospektiven (Solo-Reflektion):}

\textbf{Sprint 1-2: Grundlagen-Phase}
\begin{itemize}
    \item \textbf{Lernkurve:} MediaPipe-TouchDesigner-Integration steiler als erwartet
    \item \textbf{Stakeholder-Alignment:} Regelmäßige Kommunikation essentiell für Richtung
    \item \textbf{Technische Validation:} Komparative Tests bestätigen Hybrid-Ansatz-Wert
\end{itemize}

\textbf{Sprint 3-4: Architektur-Phase}
\begin{itemize}
    \item \textbf{Komplexitäts-Management:} Vereinfachung führt zu besserer Performance
    \item \textbf{Modularität:} Containerization erleichtert Debugging deutlich
    \item \textbf{Scope-Flexibilität:} Bereitschaft zu Architektur-Änderungen erweist sich als nützlich
\end{itemize}

\textbf{Sprint 5-6: Produktions-Phase}
\begin{itemize}
    \item \textbf{Lösungsfindung unter Druck:} Infrarot-Adaptation entsteht aus Produktionsnotwendigkeit
    \item \textbf{Stakeholder-Coordination:} Enge Abstimmung für Studio-Integration kritisch
    \item \textbf{Performance-Validation:} Real-world Testing unerlässlich für Produktionsreife
\end{itemize}