% Was funktioniert (und was nicht)

Das M.A.S.K.-System wurde erfolgreich in einer echten Filmproduktion eingesetzt. Hier die Ergebnisse:

\subsection{Was gut funktioniert hat}

\textbf{Infrarot-Tracking:} Funktioniert wie erhofft. Beamerbeleuchtung stört nicht mehr, MediaPipe erkennt Skelette zuverlässig im IR-Stream.

\textbf{64-Spike-System:} Reagiert präzise auf Hand- und Fußbewegungen. 5,625° Winkelauflösung ist ausreichend für choreographische Nuancen.

\textbf{Modularer Aufbau:} TouchDesigner-Container lassen sich einfach zwischen Projekten austauschen. Parallele Entwicklung verschiedener Visualisierungen war möglich.

\subsection{Die drei Visualisierungsmodi}

\textbf{Hand-Feuer-Effekte:}
\begin{itemize}
    \item Blaue Partikel folgen den Händen in Echtzeit
    \item Funktioniert nach initialem Setup fehlerfrei
    \item ParticleGPU-Integration für flüssige Performance
\end{itemize}

\textbf{Adaptive Kopfpartikel:}
\begin{itemize}
    \item Partikel um den Kopf, wechseln bei Handbewegungen über Schulter
    \item \textit{Problem:} Frontale Kamera-Beamer-Konstellation erfordert ständige Nachkalibrierung
    \item Funktional, aber arbeitsintensiv
\end{itemize}

\textbf{Radiales Spike-System:}
\begin{itemize}
    \item Kreis um Performer mit extremitäten-responsiven Segmenten
    \item \textit{Optimale Performance} bei Top-Down-Aufbau
    \item Vollständig fehlerfrei während gesamter Aufnahme
\end{itemize}

\subsection{Produktionstest im Albrecht-Ade-Studio}

\textbf{Tag 1 - Interaktive Aufnahmen:}
\begin{itemize}
    \item Infrarot-Modi: Problemlos
    \item RGB-Modi: Funktional, aber Kalibrierungsaufwand
    \item Setup-Wechsel zwischen Modi: <30 Minuten
    \item Systemstabilität: Keine kritischen Ausfälle
\end{itemize}

\textbf{Performance-Metriken:}
\begin{itemize}
    \item Top-Down Infrarot: 99\% Tracking-Genauigkeit
    \item Frontale RGB: Variable, abhängig von Lichtbedingungen
    \item Hand-Feuer: Robust nach Setup
\end{itemize}

\subsection{Code und Verfügbarkeit}

\textbf{Sieben Python-Skripte:}
\begin{itemize}
    \item Debug-Visualisierung mit Confidence-Mapping
    \item Koordinatentransformation für TouchDesigner
    \item ParticleGPU-Integration für Hand-Tracking
    \item Zustandsmaschine für Kopfpartikel
    \item 64-Spike Winkel-zu-Phase-System
    \item Geometrische Analyse (Distanzen, Winkel)
    \item Legacy Kalman-Filter (nicht verwendet, aber dokumentiert)
\end{itemize}

\textbf{GitHub-Repository:} Vollständiger Quellcode, TouchDesigner-Projektdateien, Installationsanweisungen.

\subsection{Praktischer Nutzen}

\textbf{Für Bildungseinrichtungen:}
Consumer-Hardware (Kinect V2 ~90€) macht Motion-Capture erschwinglich. Open-Source bedeutet: verstehen, modifizieren, erweitern.

\textbf{Für unabhängige Künstler:}
Alternative zu teuren Profi-Systemen. Funktioniert ohne Marker oder spezielle Beleuchtung.

\textbf{Für Entwickler:}
Modulare Architektur als Basis für eigene Experimente mit anderen Sensoren oder ML-Modellen.

\subsection{Grenzen des Systems}

\textbf{Was nicht so gut funktioniert:}
\begin{itemize}
    \item Schwachlicht-Performance bei RGB-Modi
    \item Steile Kamerawinkel erfordern manuelle Kalibrierung
    \item Extreme Okklusionen brauchen manchmal manuellen Override
\end{itemize}

\textbf{Was noch entwickelt werden könnte:}
\begin{itemize}
    \item Automatische Kalibrierung für variable Setups
    \item Bessere Interpolation bei Tracking-Aussetzern
    \item Integration zusätzlicher Sensoren für Hybrid-Ansätze
\end{itemize}

\subsection{Methodische Erkenntnisse}

\textbf{Agile Entwicklung mit Kunstpartnern:} Funktioniert, erfordert aber kontinuierliche Kommunikation und Flexibilität bei Änderungen.

\textbf{Keep it simple:} Der ursprüngliche Kalman-Filter-Hybrid-Ansatz war unnötig komplex. MediaPipe allein reicht aus.