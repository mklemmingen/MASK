\subsubsection{Systemarchitektur-Refactoring}

Sprint 4 markierte eine fundamentale Neuausrichtung der M.A.S.K.-Architektur: Vollständige Modularisierung von MediaPipe als primäres Tracking-System und Elimination der Kinect-Abhängigkeit.

\textbf{Technologische Konsolidierung:}
\begin{itemize}
    \item \textbf{MediaPipe-Modularisierung:} Vollständige Kapselung in eigenständige TouchDesigner-Komponenten
    \item \textbf{Kinect-Deprecation:} Entfernung aller Kinect-Dependencies aufgrund limitierter Partial-Body-Detection
    \item \textbf{Performance-Optimierung:} MediaPipe zeigt gute Robustheit bei unvollständigen Skeleton-Daten
\end{itemize}

\subsubsection{Visual-Effekt-Entwicklung}

\textbf{Bubble-Displacement-Algorithmus:}
Mehrere Implementierungsansätze für körperresponsive Bubble-Bewegungen wurden evaluiert:

\begin{enumerate}
    \item \textbf{C\# Shader Implementation:}
    \begin{itemize}
        \item Custom Compute Shader für GPU-accelerated Bubble Physics
        \item \textbf{Limitation:} Unzureichende Feedback-Loops für persistente Bubble-Lifecycles
        \item \textbf{Performance:} Hohe Framerate, aber instabile Long-term Behavior
    \end{itemize}
    
    \item \textbf{ParticleGPU Manipulation:}
    \begin{itemize}
        \item TouchDesigner-native GPU-Particle-System Customization
        \item \textbf{Limitation:} Begrenzte Kontrolle über individuelle Particle-Trajectories
        \item \textbf{Feedback-Problem:} Kurze Bubble-Lebensdauer verhinderte choreographische Integration
    \end{itemize}
    
    \item \textbf{Hybrid CPU-GPU Approach:} (Sprint 5 Development)
    \begin{itemize}
        \item Kombination aus CPU-basierter Logik und GPU-Rendering
        \item Erweiterte Lifecycle-Management für choreographische Anforderungen
    \end{itemize}
\end{enumerate}

\subsubsection{Dateninfrastruktur-Implementation}

\textbf{Vollständige DAT-Extraktion:}
Sämtliche MediaPipe-Skeleton-Nodes wurden in strukturierte \texttt{DAT}-Tabellen extrahiert:

\begin{itemize}
    \item \textbf{Node-Koordinaten:} X/Y/Z-Positionsdaten mit Zeitstempel
    \item \textbf{Confidence-Werte:} ML-Model Prediction Reliability per Node
    \item \textbf{Geometrische Parameter:} Real-time Distance/Angle Calculations
\end{itemize}

\textbf{Algorithmus-Pipeline:}

\begin{algorithm}[H]
\caption{MediaPipe-DAT Processing Pipeline}\label{alg:mediapipe_dat}
\begin{algorithmic}[1]
    \State $\text{raw\_data} \leftarrow \text{MediaPipe.extractSkeletonNodes()}$
    \State $\text{dat\_table} \leftarrow \text{structureToDAT}(\text{raw\_data})$
    \For{$\text{node} \in \text{dat\_table}$}
        \If{$\text{node.confidence} > \text{CONFIDENCE\_THRESHOLD}$}
            \State $\text{filtered\_data.append}(\text{node})$
        \Else
            \State $\text{applyInterpolation}(\text{node}, \text{previous\_frames})$
        \EndIf
    \EndFor
    \State $\text{distances} \leftarrow \text{calculateDistances}(\text{filtered\_data})$
    \State $\text{angles} \leftarrow \text{calculateAngles}(\text{filtered\_data})$
    \State $\text{smoothed\_data} \leftarrow \text{applyKalmanFilter}(\text{filtered\_data})$
    \Return $\text{smoothed\_data}, \text{distances}, \text{angles}$
\end{algorithmic}
\end{algorithm}

\subsubsection{Debug-Visualisierung}

\textbf{Real-time Feedback System:}
Implementation eines visuellen Debug-Systems für Live-Tracking-Validation:

\begin{itemize}
    \item \textbf{Video-Loopback:} RGB-Stream mit Overlay-Rendering
    \item \textbf{Node-Visualization:} Rote Blob-Overlays auf detektierten Skeleton-Points
    \item \textbf{Confidence-Mapping:} Blob-Größe korreliert mit Detection-Confidence
    \item \textbf{Exception-Handling:} Graceful Degradation bei Low-Confidence Nodes
\end{itemize}

\begin{figure}[H]
    \centering
    % \includegraphics[width=0.9\textwidth]{images/debug_visualization.png}
    \caption{Debug-Visualisierung: MediaPipe Skeleton-Nodes mit Confidence-Overlays}
    \label{fig:debug_viz}
\end{figure}

\subsubsection{Containerization und Modularität}

\textbf{TouchDesigner-Komponenten-Architektur:}
\begin{itemize}
    \item \textbf{MediaPipe-Container:} Autonome Tracking-Komponente
    \item \textbf{Data-Processing-Module:} Kalman-Filter, Distance/Angle Calculations
    \item \textbf{Visual-Generator-Containers:} Modulare Effekt-Systeme
    \item \textbf{Debug-Interface-Container:} Real-time Monitoring und Validation
\end{itemize}

\subsubsection{Performance-Metriken}

\textbf{Tracking-Performance:}
\begin{itemize}
    \item \textbf{Framerate:} Stabile hohe Bildrate bei Full-Resolution Processing
    \item \textbf{Latenz:} Sehr niedrige End-to-End-Zeit (Capture zu Visual-Output)
    \item \textbf{Robustheit:} Zuverlässige Skeleton-Detection bei guten Lichtverhältnissen
    \item \textbf{Partial-Body-Tracking:} Gute Performance bei partieller Körper-Okklusion
\end{itemize}

\subsubsection{Sprint 5 Deliverables}

\textbf{Identifizierte Herausforderungen:}
\begin{itemize}
    \item Bubble-Lifecycle-Management für Extended Choreography
    \item Performance-Optimierung für Complex Multi-Layer Visuals
    \item Beamer-Calibration für Precise Spatial Mapping
\end{itemize}

\textbf{Nächste Entwicklungsphase:}
\begin{itemize}
    \item Hybrid Bubble-Physics Implementation
    \item Studio-Integration und Hardware-Testing
    \item Choreography-specific Trigger-Logic Refinement
\end{itemize}