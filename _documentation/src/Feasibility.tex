Einschätzung der Machbarkeit des M.A.S.K.-Projekts (Stand: März 2025) unter Anwendung des TELOS-Frameworks nach Hall (2013)\footnote{Hall, J.A. \textit{Accounting Information Systems, 8th Edition}. Brooks/Cole, 2013.}.

\subsection{Technologisch}

\textbf{Technische Realisierbarkeit:}
Das Projekt ist technisch umsetzbar. Die Kernkomponenten TouchDesigner, MediaPipe und Kinect V2 sind etablierte Technologien mit dokumentierter Kompatibilität. Studien wie Babouras et al. (2024)\footnote{Babouras, A., Abdelnour, P., Fevens, T. et al. „Comparing novel smartphone pose estimation frameworks with the Kinect V2 for knee tracking during athletic stress tests." \textit{International Journal of Computer Assisted Radiology and Surgery} 19, 1321–1328 (2024).} bestätigen die Zuverlässigkeit der Kinect V2-MediaPipe-Kombination für bewegungsbasierte Anwendungen.

\textbf{Technologie-Zugang:}
Das interdisziplinäre Team verfügt über die erforderliche Hardware (Kinect V2) und nutzt TouchDesigner 2023.11 in der kostenlosen Bildungsversion. Die Filmakademie Ludwigsburg stellt zusätzliche Studioinfrastruktur bereit.

\textbf{Stakeholder-Akzeptanz:}
Die Technologiewahl wurde gemeinsam mit der Filmakademie getroffen, wodurch hohe Akzeptanz und aktive Unterstützung gewährleistet sind.

\subsection{Economic}

\textbf{Finanzierung:}
Das Projekt wird im Rahmen des regulären Studiums durchgeführt, ohne externe Finanzierung. Die einzigen Kosten entstehen durch die Kinect V2-Hardware (ca. 90€), die privat beschafft wurde. TouchDesigner und MediaPipe sind kostenfrei verfügbar.

\subsection{Legal}

\textbf{Lizenzrechtliche Aspekte:}
\begin{itemize}
    \item MediaPipe: Apache License 2.0\footnote{\url{https://github.com/google-ai-edge/mediapipe/blob/master/LICENSE}}
    \item Kinect V2 (libfreenect2): Open-Source-Lizenz\footnote{\url{https://zenodo.org/records/50641}}
    \item TouchDesigner: Nicht-kommerzielle Bildungslizenz\footnote{\url{https://derivative.ca/UserGuide/Licensing}}
\end{itemize}

Alle verwendeten Technologien sind für nicht-kommerzielle Bildungsprojekte frei nutzbar.

\subsection{Operational}

\textbf{Implementierungsanforderungen:}
\begin{itemize}
    \item Entwicklung einer synchronisierten Datenfusion zwischen Kinect V2 und MediaPipe
    \item Implementation regelbasierter Trigger für Visualisierungssteuerung
    \item Integration in die bestehende TouchDesigner-Pipeline der Filmakademie
    \item Koordination zwischen beiden Standorten (Reutlingen/Ludwigsburg)
\end{itemize}

\textbf{Projektmanagement:}
Regelmäßige Online-Meetings und punktuelle Präsenztreffen ermöglichen effektive Zusammenarbeit zwischen den Projektteams.

\subsection{Scheduling}

\textbf{Zeitplanung:}
\begin{itemize}
    \item \textbf{Projektlaufzeit:} Januar bis Mai 2025
    \item \textbf{Prototyp-Ziel:} Mitte April 2025
    \item \textbf{Finaler Einsatz:} Filmproduktion Mai 2025
    \item \textbf{Ressourcenallokation:} 2 feste Projekttage pro Woche
\end{itemize}

\textbf{Risikomanagement:}
Der modulare Entwicklungsansatz ermöglicht eine stufenweise Implementierung und reduziert das Risiko von Zeitüberschreitungen. Parallele Studienverpflichtungen wurden durch Workload-Anpassungen in anderen Modulen kompensiert.