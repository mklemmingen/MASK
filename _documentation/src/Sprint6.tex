\subsubsection{Production-Critical Optimierungen}

Sprint 6 fokussierte auf die Lösung produktionskritischer Herausforderungen, die während der Pre-Production-Tests identifiziert wurden. Alle Optimierungen entstanden als direkte Reaktion auf Stakeholder-Feedback und technische Limitationen im Filmstudio-Setup.

\subsubsection{Spike-System-Präzisionsverbesserung}

\textbf{Problem-Identifikation:}
Das ursprüngliche 8-Spike-System zeigte unzureichende Granularität für subtile Hand-/Fuß-Bewegungen der Choreographie.

\textbf{Technische Lösung:}
\begin{itemize}
    \item \textbf{Spike-Resolution-Upgrade:} 8-Spike → 64-Spike-System
    \item \textbf{Enhanced Snapping-Algorithm:} Präzisere Angle-to-Position-Mapping
    \item \textbf{Parameterizable Sensitivity:} Distance-Threshold-Control für frühere Spike-Aktivierung
\end{itemize}

\begin{algorithm}[H]
\caption{Enhanced Radial Spike Generation (64-Spike System)}\label{alg:enhanced_spike}
\begin{algorithmic}[1]
    \State $\text{SPIKE\_COUNT} \leftarrow 64$
    \State $\text{ANGLE\_RESOLUTION} \leftarrow 2\pi / \text{SPIKE\_COUNT}$
    \State $\text{DISTANCE\_THRESHOLD} \leftarrow \text{getParameterValue}(\text{"sensitivity"})$
    \State $\text{extremity\_positions} \leftarrow \text{[hands, feet]}$
    \For{$\text{pos} \in \text{extremity\_positions}$}
        \State $\text{distance} \leftarrow \text{calculateDistance}(\text{pos}, \text{screen\_center})$
        \If{$\text{distance} > \text{DISTANCE\_THRESHOLD}$}
            \State $\text{raw\_angle} \leftarrow \text{atan2}(\text{pos.y}, \text{pos.x})$
            \State $\text{spike\_index} \leftarrow \text{round}(\text{raw\_angle} / \text{ANGLE\_RESOLUTION})$
            \State $\text{spike\_intensity} \leftarrow \text{mapRange}(\text{distance}, \text{threshold}, \text{max\_distance}, 0, 1)$
            \State $\text{activateSpike}(\text{spike\_index}, \text{spike\_intensity})$
        \EndIf
    \EndFor
\end{algorithmic}
\end{algorithm}

\textbf{Parameterisierung:}
\begin{itemize}
    \item \textbf{Distance Sensitivity:} Adjustable threshold für frühere Spike-Triggering
    \item \textbf{Angular Precision:} 5.625° pro Spike (360°/64) für feinste Bewegungsauflösung
    \item \textbf{Intensity Mapping:} Lineare/Exponential-Curves für Natural-Movement-Response
\end{itemize}

\subsubsection{Beleuchtungs-Interferenz-Lösung}

\textbf{Problem-Identifikation:}
Beamer-Projektionen auf den Performer verursachten RGB-Kamera-Überbelichtung und degradierten MediaPipe-Tracking-Performance erheblich.

\textbf{Infrarot-basierte Lösung:}
\begin{itemize}
    \item \textbf{Hardware-Integration:} Kinect V2 Infrarot-Stream als Primary Input
    \item \textbf{OBS Virtual Camera:} Kinect Plugin für IR-Stream-Capture
    \item \textbf{MediaPipe-Adaptation:} IR-optimierte Skeleton-Detection
\end{itemize}

\textbf{Technische Implementation:}
\begin{enumerate}
    \item \textbf{Kinect IR-Stream-Extraction:}
    \begin{itemize}
        \item Kinect V2 SDK: Direct Infrared-Frame-Access
        \item Resolution: Hochauflösend mit hoher Bildrate
        \item Bit-Depth: 16-bit IR-Intensity-Data
    \end{itemize}
    
    \item \textbf{OBS-Plugin-Pipeline:}
    \begin{itemize}
        \item Custom Kinect-OBS-Plugin für IR-Stream-Integration
        \item Virtual Camera Output für TouchDesigner-Input
        \item Real-time IR-to-Grayscale-Conversion
    \end{itemize}
    
    \item \textbf{MediaPipe-Optimization:}
    \begin{itemize}
        \item IR-spezifische Model-Tuning
        \item Enhanced Contrast-Processing für IR-Data
        \item Maintained Skeleton-Detection-Accuracy trotz Monochrome-Input
    \end{itemize}
\end{enumerate}

\textbf{Performance-Verbesserungen:}
\begin{itemize}
    \item \textbf{Tracking-Robustheit:} Zuverlässige Skeleton-Detection unter High-Intensity-Lighting
    \item \textbf{Beleuchtungs-Immunität:} Vollständige Unabhängigkeit von RGB-Lighting-Conditions
    \item \textbf{Latenz-Reduktion:} IR-Processing schneller als RGB-Pipeline
\end{itemize}

\subsubsection{Spatial-Calibration-System}

\textbf{Problem-Identifikation:}
Misalignment zwischen Beamer-Projection und Kinect-Perspective sowie Camera-Team-Positioning erforderte flexible Spatial-Offset-Correction.

\textbf{Parametric Offset-System:}
\begin{itemize}
    \item \textbf{X/Y-Translation-Parameters:} Real-time Coordinate-Shift-Control
    \item \textbf{Rotation-Compensation:} Angular-Offset für Camera-Angle-Correction
    \item \textbf{Scale-Adjustment:} Distance-based Size-Compensation
\end{itemize}

\begin{algorithm}[H]
\caption{Spatial Calibration Pipeline}\label{alg:spatial_calibration}
\begin{algorithmic}[1]
    \State $\text{offset\_x} \leftarrow \text{getParameter}(\text{"calibration\_x"})$
    \State $\text{offset\_y} \leftarrow \text{getParameter}(\text{"calibration\_y"})$
    \State $\text{rotation\_offset} \leftarrow \text{getParameter}(\text{"rotation\_correction"})$
    \State $\text{scale\_factor} \leftarrow \text{getParameter}(\text{"scale\_adjustment"})$
    \For{$\text{node} \in \text{skeleton\_nodes}$}
        \State $\text{corrected\_x} \leftarrow (\text{node.x} + \text{offset\_x}) \times \text{scale\_factor}$
        \State $\text{corrected\_y} \leftarrow (\text{node.y} + \text{offset\_y}) \times \text{scale\_factor}$
        \State $\text{rotated\_coords} \leftarrow \text{rotatePoint}(\text{corrected\_x}, \text{corrected\_y}, \text{rotation\_offset})$
        \State $\text{updateNodePosition}(\text{node}, \text{rotated\_coords})$
    \EndFor
\end{algorithmic}
\end{algorithm}

\textbf{Calibration-Interface:}
\begin{itemize}
    \item \textbf{Real-time Parameter-Control:} Live-Adjustment während Setup-Phase
    \item \textbf{Preset-System:} Speicherbare Calibration-Profiles für verschiedene Studio-Setups
    \item \textbf{Visual-Feedback:} Grid-Overlay für präzise Alignment-Verification
\end{itemize}

\subsubsection{System-Integration und Testing}

\textbf{Pre-Production-Validierung:}
\begin{itemize}
    \item \textbf{Studio-Setup-Tests:} Vollständige Integration aller drei Optimierungen
    \item \textbf{Choreography-Rehearsal:} Live-Testing mit Performer unter Production-Conditions
    \item \textbf{Technical-Dress-Rehearsal:} Final System-Validation vor Filming
\end{itemize}

\textbf{Performance-Metriken nach Optimierung:}
\begin{itemize}
    \item \textbf{Spike-Precision:} 8x höhere Angular-Resolution
    \item \textbf{Lighting-Immunity:} Robuste Tracking-Performance unter extremer Beleuchtung
    \item \textbf{Spatial-Accuracy:} Minimale Deviation nach Calibration
    \item \textbf{System-Latency:} Sehr niedrige End-to-End-Zeit (IR-Capture zu Visual-Output)
\end{itemize}

\subsubsection{Delivery Vorbereitung}

\textbf{Filming-Ready-Status:}
\begin{itemize}
    \item Alle kritischen Performance-Issues gelöst
    \item Robustes Calibration-System implementiert
    \item IR-basierte Tracking-Pipeline produktionstauglich
    \item 64-Spike-System choreographie-ready
\end{itemize}

\textbf{Final Production-Deliverables:}
\begin{itemize}
    \item Technical-Rider für Studio-Equipment-Setup
    \item Calibration-Procedures für On-Set-Alignment
    \item Emergency-Fallback-Procedures für Technical-Issues
    \item Performance-Monitoring-Dashboard für Live-Production
\end{itemize}