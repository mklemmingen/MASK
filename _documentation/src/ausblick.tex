\subsection{Weiterentwicklungspotential und Anwendungsfelder}

M.A.S.K. funktioniert für seinen spezifischen Anwendungsfall, bietet aber Potential für Weiterentwicklungen durch andere.

\textbf{Naheliegende technische Erweiterungen:}

\textbf{Hardware-Integration:}
Die modulare Architektur könnte andere Sensoren integrieren:
\begin{itemize}
    \item \textbf{Mehrere Kinects:} Für bessere Rundumerfassung
    \item \textbf{Andere IR-Kameras:} Falls Kinect V2 nicht verfügbar
    \item \textbf{Smartphone-Kameras:} MediaPipe läuft auch auf Handys
\end{itemize}

\textbf{Software-Verbesserungen:}
Ansätze, die auf der bestehenden Lösung aufbauen könnten:
\begin{itemize}
    \item \textbf{Automatische Kalibrierung:} Weniger manuelle Einstellung bei Setup-Wechseln
    \item \textbf{Bessere Interpolation:} Glattere Bewegungen bei Tracking-Aussetzern
    \item \textbf{Mehr Visualisierungen:} Community könnte TouchDesigner-Module beitragen
\end{itemize}

\textbf{Mögliche Anwendungsfelder:}

Basierend auf dem, was funktioniert hat:

\textbf{Ähnliche Performance-Projekte:}
\begin{itemize}
    \item \textbf{Andere Tanzproduktionen:} Mit ähnlichen Beleuchtungsherausforderungen
    \item \textbf{Theaterproduktionen:} Wo Marker-basierte Systeme unpraktisch sind
    \item \textbf{Installation-Kunst:} Mit begrenzten Budgets
\end{itemize}

\textbf{Bildungsbereich:}
\begin{itemize}
    \item \textbf{Andere Hochschulen:} Mit ähnlichen Projekten und Equipment
    \item \textbf{Tanzschulen:} Für experimentelle Projekte
    \item \textbf{Maker-Spaces:} Als Demonstrator für Motion-Capture
\end{itemize}

\textbf{Was fehlt noch:}

Ehrliche Einschätzung der Limitierungen:
\begin{itemize}
    \item \textbf{Multi-Person-Tracking:} Aktuell nur eine Person zuverlässig
    \item \textbf{Extreme Lichtverhältnisse:} RGB-Modi brauchen noch Arbeit
    \item \textbf{Setup-Automatisierung:} Noch zu viel manuelle Kalibrierung
    \item \textbf{Performance-Overhead:} TouchDesigner + Python hat Verbesserungspotenzial bei custom scripts gegenüber nativen Nodes
\end{itemize}

\textbf{Community-Potential:}

Das Open-Source-Repository könnte interessant sein für:
\begin{itemize}
    \item \textbf{TouchDesigner-Community:} Als Motion-Capture-Referenz
    \item \textbf{Andere Studenten:} Mit ähnlichen Projekten
    \item \textbf{DIY-Motion-Capture-Interessierte:} Als Ausgangspunkt
\end{itemize}

Die Infrarot-MediaPipe-Pipeline ist vermutlich der wertvollste Beitrag - ein einfacher Fix, der aber ein echtes Problem löst.